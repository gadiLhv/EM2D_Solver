\subsubsection{Dielectrics}
\label{ssubsec:dielectrics}
Here, $\gamma = 0$, 
\begin{equation}
\label{eq:dielectric alpha}
\alpha_x = \alpha_y = 
\begin{cases}
\frac{1}{\epsilon_r}\,, & TE \\
\frac{1}{\mu_r}\,, & TM
\end{cases} 
\end{equation}
and $\beta = 0$


\subsubsection{Lossy Metals}
\label{ssubsec:lossy_metals}
For a metallic face, the internal fields do not need to be solved. In such cases (in a RF regime, rather than optical), the edges of the face can be teated as a surface impedance of
\begin{equation}
\label{eq:surface impedance for lossy metals}
Z_s \approx 
\begin{cases}
\brr{1 + j}
\sqrt{
	\frac{\epsilon_0}
	{2\pi f}\sigma
} \,, &  TE \\
\frac{1}{\brr{1 + j}
\sqrt{
	\frac{\epsilon_0}
	{2\pi f}\sigma}
} \,,&  TM
\end{cases},
\end{equation}
where $\sigma$ is the metallic conductivity, given in $\omega/m$ and $f$ is the current resolved frequency.


The rest of the coefficients for the matrix are then given by $\alpha_x = \alpha_y = 1$, $\beta = 0$ and
\begin{equation}
\label{eq:Lossy metal gamma coeff}
\gamma = 
\begin{cases}
-j k_0 \epsilon_r Z_s\,, & TE \\
-j k_0 \mu_r Z_s\,, & TM
\end{cases}
\end{equation}


