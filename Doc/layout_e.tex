% %
% LAYOUT_E.TEX - Short description of REFMAN.CLS
%                                       99-03-20
%
%  Updated for REFMAN.CLS (LaTeX2e)
%
\documentclass[twoside,a4paper]{refart}
\usepackage{makeidx}
\usepackage{ifthen}
\usepackage{amsmath,amssymb} 			% Additional math sym, eq
% ifthen wird vom Bild von N.Beebe gebraucht!

\def\bs{\char'134 } % backslash in \tt font.
\newcommand{\ie}{i.\,e.,}
\newcommand{\eg}{e.\,g..}
\DeclareRobustCommand\cs[1]{\texttt{\char`\\#1}}

\title{EM2D-Solver}
\author{Gadi Lahav and Darren Engwirda}

\date{}
\emergencystretch1em  %

\pagestyle{myfootings}
\markboth{Changing the layout with \textrm{\LaTeX}}%
         {Changing the layout with \textrm{\LaTeX}}

\makeindex 

\setcounter{tocdepth}{2}

%%%%%%%%%%%%%%%%%
% Abbreviations %
%%%%%%%%%%%%%%%%%

\newcommand{\fem}{\textbf{FEM}}
\newcommand{\fd}{\textbf{FD}}
\newcommand{\te}{\textbf{TE}}
\newcommand{\tm}{\textbf{TM}}
\newcommand{\bc}{\textbf{BC}}

%%%%%%%%%%%%%%%%%
% Abbreviations %
%%%%%%%%%%%%%%%%%

%%%%%%%%%%%%%%%%%
% Math commands %
%%%%%%%%%%%%%%%%%

\newcommand{\Ef}{\mathbf{E}}
\newcommand{\Hf}{\mathbf{H}}
\newcommand{\Df}{\mathbf{D}}
\newcommand{\Bf}{\mathbf{B}}
\newcommand{\Jc}{\mathbf{J}}

\newcommand{\er}{{\epsilon_rs}}
\newcommand{\mr}{{\mu_r}}

\newcommand{\dx}{\partial_x}
\newcommand{\dy}{\partial_y}
\newcommand{\dz}{\partial_z}
\newcommand{\dn}{\partial_n}
\newcommand{\ds}{\partial_s}

\newcommand{\nx}{\hat{x}}
\newcommand{\ny}{\hat{y}}
\newcommand{\nz}{\hat{z}}
\newcommand{\nn}{\hat{n}}

% All sort of braces
\newcommand{\brr}[1]{\left({#1}\right)}
\newcommand{\brs}[1]{\left[{#1}\right]}
\newcommand{\brc}[1]{\{{#1}\}}
\newcommand{\abs}[1]{\left|{#1}\right|}

%%%%%%%%%%%%%%%%%
% Math commands %
%%%%%%%%%%%%%%%%%

\begin{document}

\maketitle

\begin{abstract}
	CEM 2D solver
\end{abstract}



\tableofcontents

\newpage


%%%%%%%%%%%%%%%%%%%%%%%%%%%%%%%%%%%%%%%%%%%%%%%%%%%%%%%%%%%%%%%%%%%%


\section{Solver}
\label{sec:Solver}

In this chapter, the equations for the general Electromagnetic problems are laid out. The \fem formualation is also presented, with application notes for possible sources and future work.

\subsection{Electromagnetic Problem}
\label{subsec:EM problem}
Maxwells Equation in Frequency Domain (\fd):
\begin{subequations}
\label{eq:Maxwells eqs}
\begin{equation}
\label{subeq:Faradays law}
\nabla\times\Ef + j\omega\Bf = 0
\end{equation}
\begin{equation}
\label{subeq:Ampers law}
\nabla\times\Bf - j\omega\Df = \Jc,
\end{equation}
With the constitutive relations
\begin{equation}
\label{subeq:Gauss law - E}
\nabla\cdot\Df = \rho
\end{equation}
\begin{equation}
\label{subeq:Gauss law - H}
\nabla\cdot\Bf = 0.
\end{equation}
\end{subequations}

In addition, the definitions for the flux densities are given by
\begin{subequations}
\label{eq:Flux densities}
\begin{equation}
\label{subeq:E flux dens}
\Df = \epsilon\Ef
\end{equation}
and
\begin{equation}
\label{subeq:H flux dens}
\Bf = \mu\Hf.
\end{equation}
\end{subequations}

For this solver, the Transverse equations used, in order to degenerate the problem into a $z$ infinite problem. The choice of the direction $\nz$ is arbitrary. Hence, the formulations will be called $z$ Transverse Electric and $z$ Transverse Magnetic and denoted by \te~and \tm, respectively.

\begin{subequations}
\label{eq:Transverse formulations}
\begin{equation}
\label{subeq:TE formulation}
\brs{
	\dx\brr{
		\frac{1}{\er}\dx \cdot
	}
	+
	\dy\brr{
		\frac{1}{\er}\dy \cdot
		}
	+ k_0^2\mr
}
H_z = 
-\dx\brr{\frac{1}{\er}J_y}
+\dy\brr{\frac{1}{\er}J_x}
\end{equation}
\begin{equation}
\label{subeq:TM formulation}
\brs{
	\dx\brr{
		\frac{1}{\mr}\dx \cdot
		}
	+
	\dy\brr{
		\frac{1}{\mr}\dy \cdot
	}
	+ k_0^2\er
}E_z
= j k_0 Z_0 J_z,
\end{equation}
\end{subequations}
where $Z_0 = \sqrt{\frac{\mu_0}{\epsilon_0}}$ and $k_0 = \omega\sqrt{\epsilon_0\mu_0}$.


Some important boundary conditions (\bc s) are:
\begin{description}
\item[Impedance \bc s]\index{Impedance bc} This type of boundary condition can be used both for actual finite-conductivity surfaces, or for thin dielectric slabs. In general, the \bc~is phrase as
\begin{subequations}
	\label{eq:Impedance BC}
	\begin{equation}
	\label{subeq:Impedance BC - TE case}
	\dn H_z = j k_0 \epsilon_{r_1} \eta H_z
	\end{equation}
	for the \te~case, and 
	\begin{equation}
	\label{subeq:Impedance BC - TM case}
	\dn E_z = j k_0 \frac{\mu_{r_1}}{\eta} \eta E_z
	\end{equation}
	for the \tm~case.
\end{subequations}

Here, $eta = \sqrt{\frac{\mu_{r_2}}{\epsilon_{r_2}}}$, where the subscripts 1 and 2 describe the non-conducting and the conducting materials, respectively.

\item[Absorbing \bc s]\index{Absorbing bc} This type of boundary condition is used for scattering problems and simulates an infinite open space. The 1\textsuperscript{st} order \bc~ is given by
\begin{subequations}
	\label{eq:Absorbing bc}
	\begin{equation}
	\label{subeq:1st order absorbing bc}
		\dn \phi^s + \brs{j k_0 + \frac{\kappa\brr{s}}{2}} \phi^s = 0
	\end{equation}
	and the 2\textsuperscript{nd} order version by 
	\begin{equation}
	\label{subeq:2nd order absorbing bc}
		\frac{\partial \phi^{sc}}{\partial \rho}
		+ \brs{
			jk_0 + \frac{1}{2\rho} - \frac{1}{8\rho^2 \brr{\frac{1}{\rho} jk_0}}
		}\phi^{sc}
		-
		\frac{1}{2\rho^2\brr{\frac{1}{\rho} jk_0}}
		\frac{\partial^2\phi}{\partial s^2}
	\end{equation}
\end{subequations}
\end{description}

Here, $1/\rho(s) = \kappa(s)$ is $s$ parametrized surface curvature. On the surface, the normal derivation $\dn$ can be approximated as radial derivation.



\subsection{2D FEM Formulation}
\label{subsec:FEM formulation}
\subsubsection{Operator and \bc}
\label{ssubsec:Operator and bc}
The general boundary value problem to be resolved is $\mathcal{L}\phi = f$, explicitly given by
\begin{equation}
\label{eq:Strum-Liouville 2D}
-\dx\brr{\alpha_x\dx\phi}
-\dy\brr{\alpha_y\dy\phi}
+ \beta \phi = f.
\end{equation}
In addition, two \bc s are defined. Dirichlet \bc
\begin{equation}
\label{eq:Dirichlet BC}
\left.{\phi}\right|_{@\Gamma_1} = p
\end{equation}
and Neumann \bc,
\begin{equation}
\label{eq:Neumann BC}
\brr{
	\alpha_x \dx\phi \nx
	+
	\alpha_y \dy\phi \ny
	}\cdot \nn
+ \gamma\phi = q \,\,\, @ \,\Gamma_2.
\end{equation}

If there is no discontinuity at an interface $\Gamma_d$, we demand that $\phi$ satisfies the continuity condition
\begin{equation}
\label{eq:Cont BC}
\phi^+ = \phi^- 
\end{equation}
and the corresponding derivative continuity (smoothness)
\begin{equation}
\label{eq:Smoothness BC}
\brr{
	\alpha_x^+ \dx\phi^+\nx
	+
	\alpha_y^+ \dy\phi^+\ny
}\cdot\nn
=
\brr{
	\alpha_x^- \dx\phi^-\nx
	+
	\alpha_y^- \dy\phi^-\ny
}\cdot\nn.
\end{equation}

\subsubsection{Basis Functions}
\label{ssubsec:Basis functins}

Inside the triangular element $e$, the linear interpolated field $\phi$ is given by

\begin{equation}
\label{eq:Field 1st order interp}
\phi^e \brr{x,y} = 
a^e + b^e x + c^e y.
\end{equation}

In order to determine the coefficients for the $e^{th}$ element, field values at the three nodes $\brr{x_j,y_j}\,;\,\,j=1,2,3$ must be known. Then, the three equations
\begin{subequations}
\label{eq:Linear interp coeff eq system}
\begin{equation}
\phi_1^e = a^e + b^e x_1^e + c^e y_1^e
\end{equation}
\begin{equation}
\phi_2^e = a^e + b^e x_2^e + c^e y_2^e 
\end{equation}
\begin{equation}
\phi_3^e = a^e + b^e x_3^e + c^e y_3^e
\end{equation}
\end{subequations}
can be used to determine the values inside the element, using the expression

\begin{subequations}
\label{eq:Explicit linear interpolation}
\begin{equation}
\phi^e\brr{x,y} = \sum\limits_{j = 1}^3
{
	N_j^e\brr{x,y}\phi_j^e
}
\end{equation}
where
\begin{equation}
N_j^e\brr{x,y} = \frac{1}{2\Delta^e}
\brr{
	a_j^e + b_j^e x + c_j^e y
}
\end{equation}
and the coefficients are given explicitly as
\begin{equation}
\begin{array}{ccc}
a_1^e = x_2^e y_2^e - y_2^e x_3^e \,;& 
b_1^e = y_2^e - y_3^e \,;& 
c_1^e = x_3^e - x_2^e \,;\\

a_2^e = x_3^e y_1^e - y_3^e x_1^e \,;& 
b_2^e = y_3^e - y_1^e \,;& 
c_2^e = x_1^e - x_3^e \,;\\

a_3^e = x_1^e y_2^e - y_1^e x_2^e \,;& 
b_3^e = y_1^e - y_2^e \,;& 
c_3^e = x_2^e - x_1^e \\
\\
& \Delta^e = \frac{1}{2} 
\abs{
\begin{matrix}
1 & x_1^e & y_1^e \\
1 & x_2^e & y_2^e \\
1 & x_3^e & y_3^e
\end{matrix}
}
= \frac{1}{2}\brr{b_1^e c_2^e - b_2^e c_1^e}.
\end{array}
\end{equation}
\end{subequations}

\subsubsection{Ritz Formulation}
\label{ssubsec:Ritz formulation}

Without boundary conditions of the 2\textsuperscript{nd} and 3\textsuperscript{rd} kind incorporated, the equation system to resolve is given by
\begin{equation}
\mathbf{K}\boldsymbol{\phi} = \mathbf{b},
\end{equation}
where $\boldsymbol{\phi}=\brc{\phi_n | i = n..N_n}$, $N_n$ being the number of nodes in the simulated domain. The $\mathbf{K}$ matrix elements are given by
\begin{equation}
K_{ij} = 
\sum\limits_{e = 1}^{N_e}{
	K_{ij}^e
},
\end{equation}
where $N_e$ is the total number of elements. Each part of the series can be calculated as
\begin{equation}
K_{ij}^e = 
\frac{1}{4\Delta^e}\brr{
	\alpha_x^e b_i^e b_j^e + \alpha_y^e c_i c_j
}
+ 
\frac{\Delta^e}{12}\beta^e\brr{1 + \delta_{ij}}.
\end{equation}
The $\mathbf{b}$ vector elements are similarly given by
\begin{equation}
b_i = \sum\limits_{e = 1}^{N_e}{
	b_{i}^e
},
\end{equation}
where each addition is given by
\begin{equation}
b_i^e = \frac{\Delta^e}{3} f^e.
\end{equation}

In case of non-vanishing $\gamma$ and $q$, namely existing Neumann \bc s, the individual segment contributions to $K$ and $b$ need to be taken into consideration, as well. In this case, the 1\textsuperscript{st} order interpolation function is given by
\begin{equation}
\phi^s = \sum\limits_{j = 1}^{2}{
	N_j^s\phi_j^s
},
\end{equation}
where $N_1^s = 1 - \xi$ and $N_2^s = \xi$, while $\xi = \brs{0,1}$. $\phi^s$ denotes the field value in the $s^{th}$ segment. The individual contributions of the $s^{th}$ segment are given by
\begin{subequations}
\begin{equation}
K_{ij}^s = \gamma^s \frac{l^s}{6} \brr{1 + \delta_{ij}},
\end{equation}
\begin{equation}
b_i^s = q^s \frac{l^s}{2}.
\end{equation}
\end{subequations} 




\subsection{Material Definitions}
\label{subsec:Material Definitions}
\subsubsection{Dielectrics}
\label{ssubsec:dielectrics}
Here, $\gamma = 0$, 
\begin{equation}
\label{eq:dielectric alpha}
\alpha_x = \alpha_y = 
\begin{cases}
\frac{1}{\epsilon_r}\,, & TE \\
\frac{1}{\mu_r}\,, & TM
\end{cases} 
\end{equation}
and $\beta = 0$


\subsubsection{Lossy Metals}
\label{ssubsec:lossy_metals}
For a metallic face, the internal fields do not need to be solved. In such cases (in a RF regime, rather than optical), the edges of the face can be teated as a surface impedance of
\begin{equation}
\label{eq:surface impedance for lossy metals}
Z_s \approx 
\begin{cases}
\brr{1 + j}
\sqrt{
	\frac{\epsilon_0}
	{2\pi f}\sigma
} \,, &  TE \\
\frac{1}{\brr{1 + j}
\sqrt{
	\frac{\epsilon_0}
	{2\pi f}\sigma}
} \,,&  TM
\end{cases},
\end{equation}
where $\sigma$ is the metallic conductivity, given in $\omega/m$ and $f$ is the current resolved frequency.


The rest of the coefficients for the matrix are then given by $\alpha_x = \alpha_y = 1$, $\beta = 0$ and
\begin{equation}
\label{eq:Lossy metal gamma coeff}
\gamma = 
\begin{cases}
-j k_0 \epsilon_r Z_s\,, & TE \\
-j k_0 \mu_r Z_s\,, & TM
\end{cases}
\end{equation}




\subsection{Radiation Boundary Conditions}
\label{subsec:Radiation BCs}
In order to process the radiation \bc s, the transformation
\begin{equation}
\label{eq:Rad BC transformation}
\begin{array}{ccc}
\frac{\partial}{\partial \rho} \rightarrow \frac{\partial}{\partial n}\,\,, & 
\frac{1}{\rho} \rightarrow \kappa\brr{s}\,\,, & 
\frac{1}{\rho^2}\frac{\partial^2}{\partial \varphi^2} \rightarrow \frac{\partial^2}{\partial s^2}
\end{array}
\end{equation}
is suggested. In addition, we substitute $\phi = \phi^{inc} + \phi^{sc}$, obtaining the formulation
\begin{equation}
\begin{array}{c}
\dn\phi + \brs{
	j k_0 + \frac{\kappa}{2} - 
	\frac{j\kappa^2}{8\brr{j\kappa - k_0}}
}\phi
-
\frac{j}{2\brr{\kappa - k_0}}\ds^2 \phi
= \\
\dn\phi^{inc} + \brs{
	j k_0 + \frac{\kappa}{2} - 
	\frac{j\kappa^2}{8\brr{j\kappa - k_0}}
}\phi^{inc}
-
\frac{j}{2\brr{\kappa - k_0}}\ds^2 \phi^{inc}.
\end{array}
\end{equation}

\begin{equation}
\frac{\partial \phi}{\partial n} + 
\brs{
	jk_0 + \frac{\kappa\brr{s}}{2}
}\phi
=
\frac{\partial \phi^{inc}}{\partial n} + 
\brs{
	jk_0 + \frac{\kappa\brr{s}}{2}
}\phi^{inc}
\end{equation}

\begin{equation}
\begin{array}{c}
\gamma = \alpha\brs{jk_0 + \frac{\kappa\brr{s}}{2}} \\ \\ 
q = \alpha\frac{\partial \phi^{inc}}{\partial n} + 
\alpha\brs{jk_0 + \frac{\kappa\brr{s}}{2}}
\end{array}
\end{equation}

\begin{equation}
\alpha\frac{\partial \phi}{\partial n} + \gamma\phi = q
\end{equation}

\begin{equation}
\alpha\frac{\partial \phi}{\partial n} + \gamma\phi = q
\end{equation}



\printindex

\end{document}
